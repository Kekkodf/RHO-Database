\section{Natural Language Sentences}
%what did we discover during the interviews? what characteristic should the system have?
%Notice that, the description here, although rich, is at high level.

The RHO Institute is in need of a complete and user-friendly information management system to handle its oncology research pipeline for the clinical studies in the hospital, which encompasses the critical functions of patient enrollment, clinical data management, and statistical analysis.

As such, the primary objective of the system is to manage and keep updated the data (both personal and clinical) of patients, the logistics of when and where the research samples are obtained and stored, and the eventual treatment proposed. Moreover, the system needs to be able to group the information based on the clinical study defined by the research team, which takes place with clinical engineers and data scientists for clinical and statistical analysis.

All the hospital's research personnel (doctors, clinical engineers, and data scientists) must be registered to the platform, providing their name, surname, e-mail address, phone number, badge number, and role in the hospital; furthermore, both clinical engineers and data scientists must provide the name of their research team.

Once a cancer patient gets admitted to the hospital, a doctor will first visit the patient, making a diagnosis and proposing a treatment.
Concurrently with the visit, the patient will be registered in the system. This operation requires the following information to be provided: name, surname, date of birth, sex, fiscal code, height and weight of the patient, date, type of the visit (or list of the dates and types if there have been other medical appointments), the diagnosis, progression of the illness, state of the patient, and finally the proposed treatment. The patient may be visited more than once by the doctor, who, at each visit, will update the patient's information. During one of these medical appointments, the doctor may take blood or tissue samples to be used for research purposes. In addition, it should be possible to see the patient's clinical history, i.e., the list of sample IDs belonging to the patient already stored in the system.

These samples will be given to clinical engineers, who will first assign each of them a unique incrementally generated identification number, the sample ID, according to the GDPR guidelines for managing clinical information. It is important to underline that a singular sample has the potential to be enrolled in multiple clinical studies. The information gathered from these studies can be either pseudonymized, which involves the removal of Personal Identifying Information (PII) such as name, surname, and city of residence, or anonymized, which implies the removal or masking of the data. The choice to pseudonymize or anonymize the data is determined by whether or not RHO's researchers conduct the study.

The research team, composed of clinical engineers and data scientists, will proceed to initiate the clinical study in a collaborative effort. The clinical study is composed of both: the biochemical analysis, which is the responsibility of the clinical engineers and consists of defining the information belonging to the chemical and biological data of the extracted samples; and the statistical analysis, performed by the data scientists on the available data. If a disease is present in the biological sample, it must be enrolled using the standard for International Classification Diseases for Oncology (ICD-O). The characteristics of the biological data that must be registered are the general features of the sample, e.g., the volume of the sample, whether it is a mutated one or not, the types of cells in the sample, and the number of mutations present, the hormone receptor status of individual cells. Moreover, the system must be able to store information about the clinical results, e.g., tumor marker proteins, oncogenes, DNA and RNA segment characteristics, and specific factors related to the objectives of the clinical study.

Once the data of a sample has been registered, the researchers will provide the location where the sample is currently being stored by specifying the ID of the storage container, such as a refrigerator or a vial. If the sample is being used for experimentation, i.e., the sample is undergoing a specific clinical study by a team of researchers, then the sample will be unavailable.

The clinical results must be made available to the data scientists of the RHO institute in their original form. However, it is equally important to ensure that external researchers receive an anonymized version of the same results. This approach ensures that data is shared ethically, with due regard for privacy concerns, while enriching the amount of data available for research in the oncology field.
